\chapter{Evaluation}

% Examiners expect to find a section addressing questions such as:
\section{Review against the requirements}
\paragraph{}
The requirements for this system were correctly identified at the beginning however, they did need to be expanded upon.The system needed to be able to record a path and then follow the same path as closely as possible. The following requirements were identified at the beginning:
\begin{itemize}
  \item The ICP Model matching program.
  \item The recording model program.
  \item The robot control program.
  \item Simulated test environment with models to scale.
\end{itemize}
The system successfully does this well but there are some limitations as discussed above. The list of requirements includes an ICP model matching program which has been created and works, a recording model program that can record a path which has been created and works, a robot controlling program that follows the recorded model as closely as possible which has also been created and works and finally a simulated test environment with real world scale model which has also been created and used for testing.

\section{Methodology}
\paragraph{}
The methodology used for this project was agile sprints. This was a good way to structure the project due to it being a system that uses real world hardware and sensors which needed the ability to quickly adapt and change parts of the original design. Due to this project using sprints, the system was improved in small iterations rather than large changes and this helped the system become more stable by building upon the previous sprints work.

\paragraph{}
Due to this project being an individual person project, keeping a written and accurate track of issues was a difficult task which was often neglected and forgotten about. This had to be revised every two to three weeks to make sure tasks and issues weren't being missed.

\paragraph{}
Dispite the issues and tasks which were difficult to remember to keep a track of due to this project being an individual project, using a form of agile was a good decicison as it enabled the project to be changed and built upon rapidly each week. This methodology helped the project progress as a whole and the project would not be as complete without it.

\section{Design}
\paragraph{}
The design was thought out and planned accordingly. Some changes were needed to the overall design in the amount of nodes and the node interactions.

\paragraph{}
The major changes that need to be made to make this system better if started again would be to use the ICP model matching algorithm as a service that can be called on rather than having it continuously running and sending the resulting transformation. By using it as a service, the whole system can be controlled by the automated driving node which would result in a hybrid control system which could choose if the robot needed to act upon updated information or continue with the path originally planned.

\paragraph{}
The hardest part of this project was translating the result from the ICP algorithm into movement which guided the robot along the correct path while recovering from any errors that might make it move slightly off course. An easier way of doing this would be to use pre-existing path planning tools to plan ahead so the system could be more robust.

\paragraph{}
The recording node was well thought out from the start and needed minimal changes to work as intended. When it was realised that the best way to record the data would be at specific distances rather than user defined due to the ease of use and reliability, it improved the overall workflow of the system drastically.

\section{Use of tools}
The tools that were used were suitable for this project however, a few extra tools could have been used to make the system more reliable and to help debug the issues that occurred.

\subsection{ROS path planning}
\paragraph{}
Possible tools that would have helped include using the ROS navigation stack\cite{ROS-NAV} to plan a path to the target location. This would have saved a lot of time in trying to get the robot steering to match the direction that the ICP algorithm was returning. It also would have resulted in a likely better coping system with misalignment and bad positioning due to turning the steering into a deliberative system which can react to errors rather than a fully reactive system.

\subsection{Visualisation of the data}
\paragraph{}
A tool that would have also helped would have been some sort of tool to visualise the data that the recording node took. This would have been a helpful debugging tool when figuring out certain behaviours instead of looking at the raw numbers. A tool such as rvis\cite{RVIS}, which is built into ROS, could help with such visualisations as it can take the data produced by the laserscan in real time and visualise it against the model.

\section{Testing}
\paragraph{}
The testing for this system has been thorough. None of the testing is automated but this is difficult to accomplish when working with a fully reactive robotic system. Tests were always ran multiple times to make sure the result is always reliable. Using testing harnesses and also recording the robot position in the simulator would have helped make the testing more thorough. These two types of testing would have also helped improve the development by giving clear results wheather changes improved the system or made it worse. With more time, these extra types of testing would be implemented and if starting again, implementing these first would be a priority.

\section{Personal performance}
\paragraph{}
Developing the system went fairly well. There were some minor roadblocks such as importing libraries through CMake and getting the direction converted into movement messages which took multiple weeks to figure out. More planning and research should have gone into control strategies to make working out the movement easier. If the project was started again, more time would need to be spent on setting up testing harnesses before creating the system so when the system is being modified, there is useful feedback that can be used to improve the system. Implemeting and testing the ICP node was done well and was likely the most thought out and researched part of the whole project. The recording node was suprisingly easy to make and was expected to take a lot more time to get working. The extra scripts that were made are very useful in improving the quality of life for the system and the state changing script was fairly useful for debugging the system even though more thorough testing would have been better. Overall the project went well with the recording data node, ICP node and extra scripts being the good parts and the automated driving node and testing should have had more time spent researching and developing.

\section{Project aims}
\paragraph{}
This project set out to get Aberystwyth universities large robots to be able to automatically load themselves into a van. The final part of testing proves that the system that has been created fulfils this aim. There are however some limitations which this system has which need to be known before using it.

\paragraph{}
These limitations are that the robot must be mostly aligned with the ramp before starting, the robot must be mostly in the right position before starting and the robot must not be too far away from the ramp before starting. If all three of these limitations are met then the system will work.


% \begin{itemize}
%    \item Were the requirements correctly identified?
%    \begin{itemize}
%      \item Yes, Patern matching and moving in relation to that
%    \end{itemize}
%    \item Were the design decisions correct?
%    \begin{itemize}
%      \item Yes, it works but could have been better
%    \end{itemize}
%    \item Could a more suitable set of tools have been chosen?
%    \begin{itemize}
%      \item Maybe ROS navigation stack be used for more precice movement.
%    \end{itemize}
%    \item How well did the software meet the needs of those who were expecting to use it?
%    \begin{itemize}
%      \item TODO: Edit this after running on the real robot
%    \end{itemize}
%    \item How well were any other project aims achieved?
%    \begin{itemize}
%      \item I learned a lot more abour ROS and how to process sensordata.
%    \end{itemize}
%    \item If you were starting again, what would you do differently?
%    \begin{itemize}
%      \item Spend more time looking up movement strategies as that was the troubling part.
%      \item Use the automated movement to control everything and the ICP node as a service rather than ICP sending messages whenever it can and automated movement only partly in control, This is for easier searching.
%    \end{itemize}
% \end{itemize}
%
% Other questions can be addressed as appropriate for a project.
%
% The questions are an indication of issues you should consider. They are not intended as a specification of a list of sections.
%
% The evaluation is regarded as an important part of the project report; it should demonstrate that you are capable not only of carrying out a piece of work but also of thinking critically about how you did it and how you might have done it better. This is seen as an important part of an honours degree.
%
% There will be good things in the work and aspects of the work that could be improved. As you write this section, identify and discuss the parts of the work that went well and also consider ways in which the work could be improved.
%
% In the latter stages of the module, we will discuss the evaluation. That will probably be around week 9, although that differs each year.
