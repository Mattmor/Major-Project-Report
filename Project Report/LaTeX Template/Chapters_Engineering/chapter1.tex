\chapter{Background \& Objectives}

% This section should discuss your preparation for the project, including background reading, your analysis of the problem and the process or method you have followed to help structure your work.  It is likely that you will reuse part of your outline project specification, but as you write this report at the end of the project you should have more to discuss.
%
% \textbf{Note}:
%
% \begin{itemize}
%    \item All of the sections and text in this example are for illustration purposes. The main Chapters are a good starting point, but the content and actual sections that you include are likely to be different.
%
%    \item Look at the document MMP\_SO8 Project Report and Technical Work \cite{ProjectReportTechicalWork} for additional guidance.
%
% \end {itemize}

\section{Background}
%What was your background preparation for the project? What similar systems did you assess? What was your motivation and interest in this project?
\paragraph{}
When the robot drives into the van, the data the LIDAR captures will be a specific set of patterns. These patterns should be the same each time the robot drives into the van because the van itself is a static environment where not a lot can change. By recording these specific patterns and matching what the robot currently sees to the patterns, the robot should be able to be guided into the van.
\paragraph{}
There are some interesting research papers about this, in particular one about the evaluation of different sensors and and how they can be used effectively to track a robot position and navigate accordingly\cite{Borenstein}. In this paper they evaluate different sensors and how they can be used effectively to track a robots position and navigate accordingly. The paper mentions model matching in relation to maps (pre-existing maps and having to build the map when running) and this points to a possible solution to the problem if the map of the area is already known.
\paragraph{}
Another relevant research paper that was looked at is about autonomous driving using odometry with the help of multiple LIDAR's to correct the odometry\cite{Akai}. The issue they tackled is the fact that using odometry on its own leads to unreliable localisation, but combining it with the use of multiple LIDAR's can result in an accurate localisation method. While interesting to see how the two can be combined to create accurate localisation, the vehicle they use this method on is travelling a lot faster than the vehicle that will be driving into the van and therefore this level of accuracy will not be needed.
\paragraph{}
An original idea on how to tackle the problem of which way to drive was to use computer vision to detect the edges of the van and calculate the middle point in which the robot then drives to. Accurate edge detection would result in two specific points being detected as the edges and the difference in distance between these two points will be able to give an indication of the rotation needed for the robot to be square with the van.
\paragraph{}
More research into model matching with the use of laserscan data came up with some lecture slides from Berkeley about scan matching\cite{Abbeel}. These slides were extremely useful and suggested a good way to match the incoming laser data to the model would be to use an Iterative Closest Point algorithm (ICP). This would result in a translation and rotation that could be used to guide the robot in the correct direction. ICP works by matching each point from an incoming source to the closest point given in a model. Once the best matching point has been found, the source point gets moved slightly closer to its matched model point. This happens to every point from the source data to the model data. Due to the need to match every point, the number of points in the source data and number of points in the model should be the same. This way of matching each point from the source to the closest point in the model and iteratively moving slowly towards usually results in the best match possible and the average distance that each point needed to move by can be calculated which will return an x distance, a y distance and a rotation angle. A visualisation of how this happens can be seen in figure TODO.
\paragraph{}
The issue with using ICP is the possibility of the algorithm taking a long time to output an answer. This is because the algorithm iterates over itself many times while closing the gap between the model scan and the source scan to get a good result. More research was therefore needed to find out if ICP can be fast enough to process in real time while the robot is moving and a suitable time limit would be 100ms due to the robot moving slowly (0.1m/s). A research paper about fast ICP\cite{Levoy} answered if ICP could be fast enough with its section on high speed ICP variants. It suggested that through certain ways, ICP could output a result for a 3D scan of an elephant within 30ms. Due to the data from the LIDAR being a 2D point-plane it should take even less time than a 3D point-cloud and therefore ICP is capable of returning a result within the time required.
\paragraph{}
The large robots that the university owns use ROS\cite{ROS} (robot operating system) exclusively. The system will therefore have to make use of ROS so that it can work on the hardware that the university provides. Research into ROS shows the architecture of programs is structured in nodes and information can be sent between nodes using messages. ROS automatically handles these message interactions and the porgrammer has to set up the publishers and listeners to get information from node to node. This way of using nodes and messages to deliver information will affect how the system is designed a lot.
\paragraph{}
Due to doing a degree in Artificial Intelligence and Robotics, a major project had to be chosen with these two topics in mind. There were several robotics related projects in which could have chosen but using the big robots that the department has seemed like the most interesting of all of these projects.  Automation is also a very interesting topic and so the motivation for working on this project was to automate a repetitive task for the department and make a system that could hopefully be used on the large robots the University has for years to come.

% \begin{itemize}
%   \item Robot likely to see a specific set of patterns when driving into/out of van.
%   \item Background prep =  year 2 robotics assignment?
%   \item Motivation/interest = Interested in robotics and locomotion with a big robot seemed cool to do. It also has a practical use for the department and will likely be used for years to come.
% \end{itemize}

\section{Analysis of requirements}
% Taking into account the problem and what you learned from the background work, what was your analysis of the problem? How did your analysis help to decompose the problem into the main tasks that you would undertake? Were there alternative approaches? Why did you choose one approach compared to the alternatives?
%
% There should be a clear statement of the objectives of the work, which you will evaluate at the end of the work.
%
% In most cases, the agreed objectives or requirements will be the result of a compromise between what would ideally have been produced and what was determined to be possible in the time available. A discussion of the process of arriving at the final list is usually appropriate.
%
% As mentioned in the lectures, think about possible security issues for the project topic. Whilst these might not be relevant for all projects, do consider if there are relevant for your project. Where there are relevant security issues, discuss how they will this affect the work that you are doing. Carry forward this discussion into relevant areas for design, implementation and testing.
\paragraph{}
Looking at what was learned from the research, the most thought out and likely best way to tackle getting the direction to drive in is using model matching with ICP. This way has been used before in robotic systems and is fast enough to work in real time on a moving vehicle. A program that takes input from the LIDAR, performs ICP model matching on the data against a model and outputs the resulting transformation is therefore needed as a requirement.
\paragraph{}
Due to the model matching only working when there is already a model to match with, a recording program is needed as a requirement which will record the data from the LIDAR while the robot is driven manually. This introduces a configuration phase in which the robot has to be driven into the van manually at least once before it can drive the path manually.
\paragraph{}
The final part of the program is taking the transformation from the output of the ICP model matching program and using it to guide the robot along the recorded path. This is the last requirement needed for a complete system.
\paragraph{}
Because this project has to run on large robots the university owns (such as idris shown in figure TODO), testing it first in a simulator before trying with the real hardware is needed. Therefore, a test environment needs to be made with all models made to scale to make the simulated environment as close to how the real world environment. This simulated test environment should include the ramps at the same size they are in the real world and a van which is at the same height as it is in the real world.
\paragraph{}
This brings the list of requirements to:
\begin{itemize}
  \item The ICP Model matching program.
  \item The recording model program.
  \item The robot control program.
  \item Simulated test environment with models to scale.
\end{itemize}

%% TODO: Finish this part off.

% \begin{itemize}
%   \item Need to match a model of the pattern to what the robot currently sees.
%   \item Research into model matching suggested ICP would be a good and fast way to get the direction needed to travel
%   \item Another way to do it is to detect the edges with the lidar and work out the center of the points and drive that direction.
%   \item Objectives of work include: Making the test environment in the simulator, Making sure the robot can drive into the van manually, Making a node that records laser data at specific distances, Making a node that compares the current laser data against the model using ICP, Making a node that takes the resulting transformation and uses it to drive the robot along the same path as recorded.
%   \item Needs to drive slowly into the van so if anything goes wrong, nothing breaks.
% \end{itemize}

\section{Process}
% You need to describe briefly the life cycle model or research method that you used. You do not need to write about all of the different process models that you are aware of. Focus on the process model that you have used. It is possible that you needed to adapt an existing process model to suit your project; clearly identify what you used and how you adapted it for your needs.
\paragraph{}
The methodology that was used for this project is closely related to being agile. A sprint\cite{SCRUM} lasts one week due to the regular meetings with the project supervisor. The sprint lasts from Saturday to Friday because the meeting with the project supervisor is on Friday morning and therefore Friday afternoon is spent re-evaluating what has been achieved in the last sprint and what needs to be achieved within the next sprint.
\paragraph{}
There are several tools used to keep track of tasks and issues within sprints. A general overview was made using trello\cite{TRELLO} which gets updated at the start of every sprint. Any issues that occurred were put into github as tracked issues and this was used to update the tasks listed on trello. The final trello overview can be seen in figure TODO.
\paragraph{}
Sprints are extremely useful when working with robots due to the ability to keep track of the original plan of things to do but also change and update the plan based on any issues or new priorities that occur.
\paragraph{}
Chapter two explains the original idea for the design before any sprints occurred and chapter three explains the final design and what needed changing as each sprint re-evaluated and iterated upon the system's design.
% \begin{itemize}
%   \item Agile sprints every week.
%   \item At the end of each sprint have a meeting with supervisor and figure out what needs to be a priority for the next sprint.
% \end{itemize}
